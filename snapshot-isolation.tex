% $Header$

\documentclass{beamer}

% This file is a solution template for:

% - Talk at a conference/colloquium.
% - Talk length is about 20min.
% - Style is ornate.



% Copyright 2004 by Till Tantau <tantau@users.sourceforge.net>.
%
% In principle, this file can be redistributed and/or modified under
% the terms of the GNU Public License, version 2.
%
% However, this file is supposed to be a template to be modified
% for your own needs. For this reason, if you use this file as a
% template and not specifically distribute it as part of a another
% package/program, I grant the extra permission to freely copy and
% modify this file as you see fit and even to delete this copyright
% notice. 


\mode<presentation>
{
	\usetheme{Warsaw}
	% or ...
	
	\setbeamercovered{transparent}
	% or whatever (possibly just delete it)
}


\usepackage[english]{babel}
% or whatever

\usepackage[latin1]{inputenc}
% or whatever

\usepackage{times}
\usepackage[T1]{fontenc}
\graphicspath{ {./graphics/} }
% Or whatever. Note that the encoding and the font should match. If T1
% does not look nice, try deleting the line with the fontenc.


\title[Analyzing Snapshot Isolation] % (optional, use only with long paper titles)
{Analyzing Snapshot Isolation}

\author[Andrea Cerone, Alexey Gotsman] % (optional, use only with lots of authors)
{Andrea Cerone \and Alexey Gotsman}
% - Give the names in the same order as the appear in the paper.
% - Use the \inst{?} command only if the authors have different
%   affiliation.

\institute[IMDEA Software Institute] % (optional, but mostly needed)
{IMDEA Software Institute}
% - Use the \inst command only if there are several affiliations.
% - Keep it simple, no one is interested in your street address.

\date[CFP 2003] % (optional, should be abbreviation of conference name)
{PODC 2016}
% - Either use conference name or its abbreviation.
% - Not really informative to the audience, more for people (including
%   yourself) who are reading the slides online

\subject{Theoretical Computer Science}
% This is only inserted into the PDF information catalog. Can be left
% out. 



% If you have a file called "university-logo-filename.xxx", where xxx
% is a graphic format that can be processed by latex or pdflatex,
% resp., then you can add a logo as follows:

% \pgfdeclareimage[height=0.5cm]{university-logo}{university-logo-filename}
% \logo{\pgfuseimage{university-logo}}



% Delete this, if you do not want the table of contents to pop up at
% the beginning of each subsection:
\AtBeginSubsection[]
{
	\begin{frame}<beamer>{Outline}
	    \tableofcontents[currentsection,currentsubsection]
    \end{frame}
}


% If you wish to uncover everything in a step-wise fashion, uncomment
% the following command: 

%\beamerdefaultoverlayspecification{<+->}


\begin{document}

\begin{frame}
	\titlepage
\end{frame}

\begin{frame}{Outline}
	\tableofcontents
\end{frame}

\section{Introduction}

\begin{frame}{Intro}
	\begin{itemize}
		\item We focus on \emph{Snapshot Isolation} presented in the last talk
		\item ... same context of DBMS and transactional memory systems
	\end{itemize}
\end{frame}

\begin{frame}{Intro}
	\begin{itemize}
		\item DBMS typically offer various guarantees for transaction management
		\item Each mode exhibits different \emph{anomalies}
		\item Stronger modes exhibit less anomalies at expense of performance
		\begin{itemize}
			\item Stronger guarantees incur more overhead on the DBMS side
			\item Less allowed behaviors $ \rightarrow $ more concurrent transactions expected to abort
		\end{itemize}
	\end{itemize}
\end{frame}

\begin{frame}{Snapshot Isolation}
	\textbf{Snapshot Isolation} - originally specified as operational model:
	\begin{itemize}
		\item When transaction $ T $ begins, $ snapshot_T $ is taken
		\item All reads in $ T $ read values from $ snapshot_T $
		\item All writes in $ T $ write to transient write-set
		\item T commits only if passes write conflict check:
		\begin{itemize}
			\item No object in $T$'s write-set was updated by other transactions since $snapshot_T$ was taken
		\end{itemize}
	\end{itemize}	
\end{frame}

%\begin{frame}{Example}
%	The operational model allows write-skew anomalies
%	\begin{figure}
%		\includegraphics[scale=0.22]{write-skew}
%	\end{figure}
%\end{frame}


\section{Snapshot Isolation}
\subsection{Definitions}
\begin{frame}{Definitions}
	We'll start by formulating a declarative definition of Snapshot Isolation.
\end{frame}

\begin{frame}
	We'll use a lot of notation similar to Daniel's talk two weeks ago
	\begin{itemize}
		\item $ Obj =  \{ x, y, \dots \} $ - objects in the data-set
		\item $ Event = \{ e, f, \dots \} $ - transaction events
		\item $ Op = \{ read(x,n), write(x, n) \mid x \in Obj, n \in \mathbb{Z}\} $ 
		\item $ op: Event \rightarrow Op $
	\end{itemize}
\end{frame}

\begin{frame}
	\begin{itemize}
		\item \textbf{Strict partial order} - transitive + irreflexive relation
		\item \textbf{Total order} - strict partial order that orders all pairs
	\end{itemize}
\end{frame}

\begin{frame}
	A \textbf{transaction} $ T, S, \dots $ is a pair $ (E, po ) $ where $ E \subseteq Event $ is a \emph{finite}, non-empty set of events and \textbf{program-order} $ po \subseteq E \times E $ is a total order.  \\
\end{frame}

\begin{frame}
	A \textbf{history} is a pair $ \mathcal{H} = (\mathcal{T},SO) $ where $\mathcal{T}$ is a finite set of transactions with disjoint set of events and the \textbf{session-order} $ SO \subseteq \mathcal{T} \times \mathcal{T} $ is a union of total orders defined on disjoint subsets of $\mathcal{T}$, which correspond to transactions in different sessions.
\end{frame}

\begin{frame}
	We elide treatment of:
	\begin{itemize}
		\item aborted transactions - all transactions in any history are committed.
		\item infinite computations - histories are always finite.
	\end{itemize}
\end{frame}

\begin{frame}
	An \textbf{abstract execution} is a tuple $ \mathcal{X} = (\mathcal{T}, SO, VIS, CO) $, where $(\mathcal{T}, SO)$ is a history and the \textbf{visibility} and \textbf{commit order} $ VIS, CO \subseteq \mathcal{T} \times \mathcal{T} $ are such that $ VIS \subseteq CO $ and $CO$ is total.
\end{frame}

\begin{frame}
	\begin{itemize}
		\item We'll use $ (T, S) \in VIS$ and $ T \overset{VIS}{\longrightarrow} S $ interchangeably for $VIS$ and other relations.
		\item For $\mathcal{H}=(\mathcal{T}, SO)$ we'll shorten $(\mathcal{T}, SO, VIS, CO)$ to $(\mathcal{H}, VIS, CO)$
	\end{itemize}
\end{frame}

\begin{frame}
	For the relations defined in abstract execution:
	\begin{itemize}
		\item $ T \overset{VIS}{\longrightarrow} S $ means that $T$ is included in $S$'s snapshot.
		\item $ T \overset{CO}{\longrightarrow} S $ means that $T$ is commited before $S$.
		\item $ VIS \subseteq CO $ makes sure that snapshots include only already committed transactions.
	\end{itemize}
\end{frame}

\begin{frame}
	For a set $ A $ and a total order $ R \subseteq A \times A $
	\begin{itemize}
		\item $max_R(A) = \{ a \mid \forall b \in A: a = b \vee (b, a) \in R \} $
		\item $min_R(A) = \{ a \mid \forall b \in A: a = b \vee (a, b) \in R \} $
		\item $R^{-1}(a) = \{ b \mid  (b, a) \in R\} $
		\item $R_1;R_2 = \{(a,b) \mid \exists c: (a,c) \in R_1 \wedge (c,b) \in R_2\} $
		\item $R? = R \cup \{ (a, a) \mid a \in A \}$
		\item $R^+$ a transitive closure of $R$
		\item $R^*$ a transitive and reflexive closure of $R$
	\end{itemize}
\end{frame}

\begin{frame}
	For $T = (E, po)$ we'll use:
	\begin{itemize}
		\item $ T \vdash write(x,n) $ if $T$ writes to $x$ and $n$ is s.t.
		$$ op\left(max_{po}\{e \mid op\left(e\right)  = write\left(x, \_ \right) \} \right) = write(x,n)$$
		\item $ T \vdash read(x,n) $ if $T$ reads $x$ before writing to is and $n$ is s.t.
		$$ op\left(min_{po}\{e \mid op\left(e\right)  = \_\left(x, \_ \right) \} \right) = read(x,n)$$
	\end{itemize}
	We'll also use $ WriteTx_x = \{ T \mid T \vdash write(x,\_)\}$
\end{frame}

\begin{frame}
	We'll now try to define \emph{snapshot isolation} and \emph{serializability} in terms of \textbf{consistency axioms}:
	\begin{equation*}
		\begin{aligned}
			ExecSI = {} &
				\begin{aligned} 
					\{ \mathcal{X} \mid \mathcal{X} \vDash & INT \wedge EXT \wedge SESSION \wedge \\
														& PREFIX \wedge NOCONFLICT \}
				\end{aligned}
				\\
			ExecSER = {} & \{ \mathcal{X} \mid \mathcal{X} \vDash INT \wedge EXT \wedge SESSION \wedge TOTALVIS \} \\
			HistSI = {} & \{ \mathcal{H} \mid \exists VIS,CO : (\mathcal{H}, VIS, CO) \in ExecSI \} \\
			HistSER = {} & \{ \mathcal{H} \mid \exists VIS,CO : (\mathcal{H}, VIS, CO) \in ExecSER \} 
		\end{aligned}
	\end{equation*}
\end{frame}

\begin{frame}
	$INT$ - \textbf{internal consistency axiom}: ensures that a read event $e$ on object $x$ returns the same value a as the last write or read on $x$ in the same transaction.
	\begin{multline*}
		\forall (E,po)\in \mathcal{T} . \forall e \in E . \forall x,n: \\
			op(e) = read(x,n) \wedge \{f\mid op(f) = \_(x,\_)\wedge f \overset{po}{\longrightarrow} e\} \ne \emptyset \Rightarrow \\
			op\left(max_{po}\{f \mid op\left(f\right) = \_ \left(x, \_\right) \wedge f \overset{po}{\longrightarrow} e\}\right) = \_(x,n)
	\end{multline*}
\end{frame}

\begin{frame}
	$EXT$ - \textbf{external consistency axiom}: ensures that if $T\vdash read(x,n)$ then the value is taken from the last visible transaction that wrote to $x$ according to commit order.
	\begin{multline*}
		\forall T \in \mathcal{T} . \forall x, n: \\
		T \vdash read(x,n) \Rightarrow
		max_{CO}\left( VIS^{-1}\left( T \right) \cap WriteTx_x \right) \vdash write(x,n)
	\end{multline*}
\end{frame}

\begin{frame}
	$SESSION$ - \textbf{session visibility} requires a snapshot to include all preceding transactions of the same session.
	$$ SO \subseteq VIS $$
\end{frame}

\begin{frame}
	\begin{figure}
		\includegraphics[scale=0.3]{fig2a}
	\end{figure}
\end{frame}

\begin{frame}
	$PREFIX$ - ensures that if snapshot taken by $T$ includes $S$, then it includes all transactions committed before $S$ as well.
	$$ CO; VIS \subseteq VIS $$
\end{frame}

\begin{frame}
	The \textbf{long-fork} anomaly is prevented by $PREFIX$ axiom:
	\begin{figure}
		\includegraphics[scale=0.3]{fig2c}
	\end{figure}
\end{frame}

\begin{frame}
	$NOCONFLICT$ - ensures that for any two transactions writing to the same object, one has to be aware of the other.
	\begin{multline*}
	\forall T,S \in \mathcal{T}. \forall x,n . \\
	\left( T,S \in WriteTx_x \wedge T \ne S \right)
	\Rightarrow
	\left( T \overset{VIS}{\longrightarrow} S \vee S \overset{VIS}{\longrightarrow} T \right)
	\end{multline*}
\end{frame}

\begin{frame}
	The \textbf{lost-update} anomaly is prevented by $NOCONFLICT$ axiom:
	\begin{figure}
		\includegraphics[scale=0.3]{fig2b}
	\end{figure}
\end{frame}

\begin{frame}
	$TOTALVIS$ - requires total order on the visibility relation, giving us \emph{serializability} of transactions.
	$$ VIS = CO $$
\end{frame}

\begin{frame}
	The \textbf{write-skew} anomaly allowed by Snapshot Isolation is prevented by $TOTALVIS$ axiom:
	\begin{figure}
		\includegraphics[scale=0.25]{fig2d}
	\end{figure}
\end{frame}

\subsection{Dependency Graphs}

\begin{frame}
	\begin{itemize}
		\item Our goal now is to characterize SI in terms of dependencies between transactions.
		\item Then we'll be able to decide whether SI allows a given history by looking for appropriate dependencies.
	\end{itemize}
\end{frame}

\begin{frame}
	Let $\mathcal{X} = (\mathcal{T}, SO, VIS, CO)$ be an execution, for $x \in Obj$ we define the following relations on $\mathcal{T}_\mathcal{X}$:
	\begin{itemize}
		\item $WR_\mathcal{X}(x)$ \textbf{read-dependency}:
		\begin{multline*}
			T \xrightarrow{WR_\mathcal{X}(x)} S \Leftrightarrow \\
			S \vdash read(x,n) \wedge T = max_{CO}\left( VIS^{-1}(S) \cap WriteTx_x \right)
		\end{multline*} \\
		Informally: T $ \xrightarrow{WR_\mathcal{X}(x)} S $ means that $S$ reads $T$'s write to $x$.
	\end{itemize}
\end{frame}

\begin{frame}
	Let $\mathcal{X} = (\mathcal{T}, SO, VIS, CO)$ be an execution, for $x \in Obj$ we define the following relations on $\mathcal{T}_\mathcal{X}$:
	\begin{itemize}
		\item $WW_\mathcal{X}(x)$ \textbf{write-dependency}:
		\begin{multline*}
		T \xrightarrow{WW_\mathcal{X}(x)} S \Leftrightarrow T \xrightarrow{CO}S \wedge T,S \in WriteTx_x
		\end{multline*} \\
		Informally: T $ \xrightarrow{WW_\mathcal{X}(x)} S $ means that $S$ overwrites $T$'s write to $x$.
	\end{itemize}
\end{frame}

\begin{frame}
	Let $\mathcal{X} = (\mathcal{T}, SO, VIS, CO)$ be an execution, for $x \in Obj$ we define the following relations on $\mathcal{T}_\mathcal{X}$:
	\begin{itemize}
		\item $RW_\mathcal{X}(x)$ \textbf{anti-dependency}:
		\begin{multline*}
		T \xrightarrow{RW_\mathcal{X}(x)} S \Leftrightarrow \\
		T \ne S \wedge \exists T^\prime . T^\prime \xrightarrow{WR_\mathcal{X}(x)}T \wedge T^\prime \xrightarrow{WW_\mathcal{X}(x)}S
		\end{multline*} \\
		Informally: T $ \xrightarrow{RW_\mathcal{X}(x)} S $ means that $S$ overwrites the write to $x$ read by $T$.
	\end{itemize}
\end{frame}

\begin{frame}
	A \textbf{dependency graph} is a tuple $\mathcal{G} = (\mathcal{T}, SO, WR, WW, RW)$, where $(\mathcal{T}, SO)$ is a history and:
	\begin{itemize}
		\item WR: $Obj \rightarrow 2^{\mathcal{T} \times \mathcal{T}}$ is such that:
		\begin{itemize}
			\item $\forall T,S. \forall x . T \xrightarrow{WR(x)}S \Rightarrow $ 
				  $	\exists n. T \ne S \wedge T \vdash write(x,n) \wedge S \vdash read(x,n)$
			\item $\forall S \in \mathcal{T}. \forall x. S \vdash read(x,\_) \Rightarrow \exists T. T \xrightarrow{WR(x)} S $
			\item $\forall T, T^\prime, S \in \mathcal{T}. \forall x. \left( T \xrightarrow{WR(x)} S \wedge T^\prime \xrightarrow{WR(x)} S \right) \Rightarrow T = T^\prime $
		\end{itemize}
	\end{itemize}
\end{frame}

\begin{frame}
A \textbf{dependency graph} is a tuple $\mathcal{G} = (\mathcal{T}, SO, WR, WW, RW)$, where $(\mathcal{T}, SO)$ is a history and:
	\begin{itemize}
		\item WW: $Obj \rightarrow 2^{\mathcal{T} \times \mathcal{T}}$ is such that for every $x \in Obj$, $WW(x)$ is a total order on the set $WriteTx_x$.
	\end{itemize}
\end{frame}


\begin{frame}
	A \textbf{dependency graph} is a tuple $\mathcal{G} = (\mathcal{T}, SO, WR, WW, RW)$, where $(\mathcal{T}, SO)$ is a history and:
	\begin{itemize}
		\item RW: $Obj \rightarrow 2^{\mathcal{T} \times \mathcal{T}}$ is derived from WR and WW as in the definition of $WR_\mathcal{X}(x)$.
	\end{itemize}
\end{frame}

\begin{frame}
	\textit{Proposition:} \\ 
	For any $\mathcal{X} \in ExecSI$,
	$$
		graph(\mathcal{X}) = (\mathcal{T}_\mathcal{X}, SO_\mathcal{X}, WR_\mathcal{X}, WW_\mathcal{X}, RW_\mathcal{X})
	$$
	is a dependency graph. \\
	
	\emph{Proof:} \\
	By showing $graph(\mathcal{X})$ satisfies all requirements of a dependency graph.
\end{frame}

\subsection{Characterization}

\begin{frame}
	We'll show that SI is characterized by dependency graphs that contain only cycles with at least two adjacent anti-dependency edges.
\end{frame}

\begin{frame}
	\textit{Theorem:} \\
	Let
	$$
	\begin{aligned}
		GraphSER = \{ \mathcal{G} \mid & \left( \mathcal{T}_\mathcal{G} \vDash INT \right) \\
		      & \left(
		      	\left(
		      	 SO_\mathcal{G} \cup WR_\mathcal{G} \cup WW_\mathcal{G} \cup RW_\mathcal{G}
		      	\right) \text{is acyclic}
		        \right) \}
	\end{aligned}
	$$
	Then 
	$$
	\begin{aligned}
		HistSER = \{ \mathcal{H} \mid & \exists WR, WW, RW. \\
		& \left( \mathcal{H}, WR, WW, RW \right) \in GraphSER \}
	\end{aligned}
	$$
	In other words, execution is serializable if it can be extended into an acyclic dependency graph.
\end{frame}


\begin{frame}
	\textit{Theorem:} \\
	Let
	$$
		\begin{aligned}
		GraphSI = 
			\{ 
				\mathcal{G} 
				\mid &
				\left( \mathcal{T}_\mathcal{G} \vDash INT \right) \wedge \\
					 & \left(
						\left(
							\left(
								SO_\mathcal{G} \cup WR_\mathcal{G} \cup WW_\mathcal{G}
							\right)  ; RW_\mathcal{G}?
					\right) \ is \ acyclic
				\right)
			\}	
		\end{aligned}
	$$
	Then
	$$
		HistSI = \{ \mathcal{H} \mid \exists WR, WW, RW. (\mathcal{H}, WR, WW, RW) \in GraphSI \}
	$$
	\\
	Similarly, $\mathcal{H} \in HistSI$ if $\mathcal{H}$ can be extended to a dependency graph satisfying $GraphSI$ 
\end{frame}


\begin{frame}
Example:
\begin{figure}
	\includegraphics[scale=0.25]{fig2d}
\end{figure}
\begin{itemize}
	\item Prohibited under \textbf{serializability}, and has a dependency graph cycle $ T_1 \xrightarrow{RW} T_2 \xrightarrow{RW} T_1 $.
	\item However, allowed under \textbf{SI}
\end{itemize}
\end{frame}

\begin{frame}
In contrast, following is \emph{not} allowed under SI:
\begin{figure}
\includegraphics[scale=0.2]{fig2b}
\includegraphics[scale=0.2]{fig2c}
\end{figure}
contain cycles without adjacent anti dependencies.
\end{frame}

\begin{frame}
	To prove it we'll show a stronger result: \\
	\begin{enumerate}
		\item \textbf{Soundness:} $ \forall \mathcal{G} \in GraphSI. \exists \mathcal{X} \in ExecSI. graph(\mathcal{X}) = \mathcal{G} $ 
		\item \textbf{Completeness:} $ \forall \mathcal{X} \in ExecSI. graph(\mathcal{X}) \in GraphSI$
	\end{enumerate}
	The \textit{completeness} closely follows from existing results\footnote{Making Snapshot Isolation Serializable, 2005, A. Fekete et al}. We will focus on the \textit{soundness}.
	
\end{frame}

\begin{frame}
Proof sketch: \\

\begin{itemize}
	\item Construct a basic \textbf{pre-execution} from $\mathcal{G}$
	\item Iteratively extend it until satisfies execution definition
\end{itemize}

\end{frame}

\begin{frame}
	A tuple $\mathcal{P} = (\mathcal{T}, SO, VIS, CO) $ is a \textbf{pre-execution} if it satisfies all the conditions of being an \emph{execution}, except $CO$ is a strong partial order that may not be total. We let $PreExecSI$ be the set of pre-executions satisfying the SI axioms:
	$$
		\begin{aligned}
			PreExecSI = \{ \mathcal{P} \mid \mathcal{P} \vDash & INT \wedge EXT \wedge SESSION \wedge \\
															   & PREFIX \wedge NOCONFLICT \}
		\end{aligned}
	$$
\end{frame}

\begin{frame}
	\begin{itemize}
		\item For a given dep. graph $\mathcal{G} = (\mathcal{H}, WR, WW, RW)$, let $\mathcal{P} = (\mathcal{H}, VIS, CO)$ a respective pre-execution.
		\item To conform with $\mathcal{G}$ we require that $VIS,CO$ hold: \\
			\begin{align}
				SO \cup WR \cup WW & \subseteq VIS \\
				CO; VIS & \subseteq VIS \\
				VIS & \subseteq CO \\
				CO; CO & \subseteq CO \\
				VIS; RW & \subseteq CO
			\end{align}
	\end{itemize}
\end{frame}

\begin{frame}
	\textit{Lemma:} \\
	Let $\mathcal{G} = (\mathcal{T}, SO, WR, WW, RW)$ be a dependency graph, for any relation $R \subseteq \mathcal{T} \times \mathcal{T}$, the relations
	$$
		\begin{aligned}
			VIS = {} & (((SO \cup WR \cup WW); RW?) \cup R)^*; \\
			         & (SO \cup WR \cup WW) \\
			 CO = {} & (((SO \cup WR \cup WW); RW?) \cup R)^+
		\end{aligned}
	$$
	are a solution to the system of inequalities in the previous slide. They also are the smallest solution to the system for which $R \subseteq CO$.
\end{frame}

\begin{frame}
	Back to the proof: \\
	Let $\mathcal{G} = (\mathcal{T}, SO, WR, WW, RW) \in GraphSI$
	\begin{itemize}
		\item Define $\mathcal{P}_0$ derived from the last lemma by fixing $R_0 = \emptyset$.
		\item Construct $\{\mathcal{P}_i = (\mathcal{T}, SO, VIS_i, CO_i)\}^n_{i=0}$ series of pre-executions.
		\item While $CO_i$ is not total:
		\begin{itemize}
			\item Pick arbitrary pair $T,S$ not ordered by $CO_i$
			\item $R_{i+1} = R_i \cup \{(T,S)\}$
			\item Use the lemma with $R = R_{i+1}$ to derive $VIS_{i+1}, CO_{i+1}$ (and thus $\mathcal{P}_{i+1}$)
		\end{itemize}
		\item Let $\mathcal{X} = \mathcal{P}_n$ as $CO_n$ is now total. $\square$
	\end{itemize}
\end{frame}

\section{Static Analysis}
\subsection{Transaction Chopping}

\begin{frame}
	\textbf{Transaction Chopping under SI:}
	\begin{itemize}
		\item We'll derive a static analysis that checks if transactions can be chopped into smaller sessions
		\item The analysis will suggest an optimized program provided any execution with chopped transactions does not exhibit new behaviors.
	\end{itemize}
\end{frame}


\begin{frame}
	For history $\mathcal{H}$, let
	$$ \approx_\mathcal{H} = SO_\mathcal{H} \cup SO^{-1}_\mathcal{H} \cup \{ (T,T) \mid T \in \mathcal{T}_\mathcal{H}\}$$
	be the equivalence relations grouping transactions from the same session.
\end{frame}

\begin{frame}
	Let $\boxed{T}_\mathcal{H} = (E, po)$ where 
	$ E = \left(\bigcup \{E_S \ mid S \approx_\mathcal{H} T \} \right) $
	and
	$$
	\begin{aligned}
		po = \{ (e,f) \mid & \left( \exists S . e,f \in E_S \wedge e \xrightarrow{po_S} f \wedge S \approx_\mathcal{H} T \right) \vee \\ 
		                   & \left( \exists S, S^\prime . e\in E_S \wedge f \in E_{S^\prime} \wedge S \xrightarrow{SO_\mathcal{H}} S^\prime \wedge S^\prime \approx_\mathcal{H} T \right) \}
	\end{aligned}
	$$
	\\
	Informally $\boxed{T}_\mathcal{H}$ is the result of splicing all transactions in session of $T$ into the same transaction.
\end{frame}

\begin{frame}
	\begin{itemize}
		\item	For history $\mathcal{H}$, let $splice(\mathcal{H})=\left( \{ \boxed{T}_\mathcal{H} \mid T \in \mathcal{T}_\mathcal{H} \}, \emptyset \right)$ history resulting from splicing all sessions in a history.
		\item $\mathcal{G} \in GraphSI $ is \textbf{spliceable} if exists a dependency graph $ \mathcal{G}^\prime \in GraphSI $ such that $ \mathcal{H}_{\mathcal{G}^\prime} = splice(\mathcal{H}_\mathcal{G})$.
		\item For graph $\mathcal{G}$ we let $\approx_\mathcal{G} = \approx_{\mathcal{H}_\mathcal{G}}$.
	\end{itemize}
\end{frame}

\begin{frame}
Example, let graph $\mathcal{G}$:
\begin{figure}
	\includegraphics[scale=0.28]{fig4}
\end{figure}
The above graph is not splice-able: $\boxed{S}_\mathcal{G}$ observes write by $\boxed{T}_\mathcal{G}$ to $acct1$ but not to $acct2$.
\end{frame}

\begin{frame}
	Given $\mathcal{G}$ let \textbf{dynamic chopping graph} $DCG(\mathcal{G})$ obtained from $\mathcal{G}$ by
	\begin{itemize}
		\item Removing $WR_\mathcal{G}, WW_\mathcal{G}, RW_\mathcal{G}$ edges between transactions related by $\approx_\mathcal{G}$
		\item Adding \textbf{predecessor} edges $SO^{-1}_\mathcal{G}$
		\item We'll call $SO$ \textbf{successor} edges
		\item And  call $\left( WR_\mathcal{G} \cup WW_\mathcal{G} \cup RW_\mathcal{G} \right) \setminus \approx_\mathcal{G}$ \textbf{conflict} edges
	\end{itemize}
\end{frame}

\begin{frame}
A cycle in $DCG(\mathcal{G})$ is \textbf{critical} if:
\begin{itemize}
	\item Does not contain 2 occurrences of the same vertex
	\item Contains 3 consecutive edges in form of \textit{conflict,predecessor,conflict}
	\item Any 2 anti dependency edges $(RW_\mathcal{G}\setminus \approx_\mathcal{G})$ are separated by at least one read $(WR_\mathcal{G}\setminus \approx_\mathcal{G})$ or write $(WW_\mathcal{G}\setminus \approx_\mathcal{G})$ dependency edges.
\end{itemize}
\end{frame}


\begin{frame}
\begin{figure}
	\includegraphics[scale=0.35]{fig5a}
\end{figure}
This example contains a critical cycle with $$.\xrightarrow{S}.\xrightarrow{WR}.\xrightarrow{P}.\xrightarrow{RW}.$$
\end{frame}


\begin{frame}
	\textit{Theorem:} \\
	For $\mathcal{G} \in GraphSI$, if $DCG(\mathcal{G})$ contains no critical cycles, then $\mathcal{G}$ is splice-able.
\end{frame}


\begin{frame}
We use the last theorem to derive the static analysis.
\begin{itemize}
	\item Assume set of \textbf{programs} $\mathcal{P}=\{P_1, P_2, \dots\}$, each defining code of a session resulting from chopping a single transaction.
	\item Each $P_i$ is composed of $k_i$ \textbf{program pieces}
	\item $W^i_j$ and $R^i_j$ sets of objects written or read by j-th piece of $P_i$ 
\end{itemize}
\end{frame}

\begin{frame}
\begin{figure}
\includegraphics[scale=0.28]{fig4}
\end{figure}
For $transfer$ session we have 2 program pieces with
\begin{itemize}
\item $W^1_1 = R^1_1 = \{acct1\}, W^1_2 = R^1_2 = \{acct2\}$
\end{itemize}
\end{frame}

\begin{frame}
	\begin{itemize}
		\item History $\mathcal{H}$ \textbf{can be produced} by programs $\mathcal{P}$ if there's 1:1 correspondence between every session in $\mathcal{H}$ and program $P_i\in\mathcal{P}$, and each transaction in the session corresponds to respective program piece, along with its read/write sets.
		\item Chopping is defined \textbf{correct} if every dependency graph $\mathcal{G}\in GraphSI$, where $\mathcal{H}_\mathcal{G}$ can be produced by $\mathcal{P}$ is splice-able.
	\end{itemize}
\end{frame}

\begin{frame}
	We check correctness of chopping $\mathcal{P}$ using it's \textbf{static chopping graph} $SCG(\mathcal{P})$.\\
	Graph's nodes are program pieces in form of $(i, j)$ and the edge $(i_1, j_1), (i_2, j_2)$ is present if:
	\begin{itemize}
		\item $i_1 = i_2$ and:
		\begin{itemize}
			\item $j_1 < j_2$ (a \textbf{successor} edge)
			\item $j_1 > j_2$ (a \textbf{predecessor} edge)
		\end{itemize}
		\item $i_1 \ne i_2$ and:
		\begin{itemize}
			\item $W^{i_1}_{j_1}\cap R^{i_2}_{j_2} \ne \emptyset$ (a \textbf{read dependency} edge)
			\item $W^{i_1}_{j_1}\cap W^{i_2}_{j_2} \ne \emptyset$ (a \textbf{write dependency} edge)
			\item $R^{i_1}_{j_1}\cap W^{i_2}_{j_2} \ne \emptyset$ (an \textbf{anti dependency} edge)
		\end{itemize}
	\end{itemize}
\end{frame}

\begin{frame}
	\begin{itemize}
		\item The edge set of static graphs $SCG(\mathcal{P})$ over-approximate the edges set of the dynamic graphs $DCG(\mathcal{G})$ for corresponding to graphs $\mathcal{G}$ produced by programs $\mathcal{P}$.
		\item The chopping defined by $\mathcal{P}$ is correct if $SCG(\mathcal{P})$ contains no critical cycles (as defined for dynamic graphs).
	\end{itemize}
\end{frame}

\begin{frame}
	\begin{figure}
		\includegraphics[scale=0.28]{fig56}
	\end{figure}
\end{frame}

\subsection{Robustness}

\begin{frame}
	\textbf{Robustness}: \\
	We'll derive an analysis that check where an application behaves the same way under a weak consistency model as it does under a strong one.
\end{frame}

\begin{frame}
	\textbf{Robustness against SI} \\
	Check if a given application running under \textbf{Serializability}, does not produce new histories when runs under \textbf{SI}... \\
	i.e. code does not produce histories in $HistSI \setminus HistSER$
\end{frame}


\begin{frame}
\textit{Theorem:} \\
For any $\mathcal{G}$, we have $\mathcal{G} \in GraphSI \setminus GraphSER $ if $\mathcal{T}_\mathcal{G} \vDash INT$, $\mathcal{G}$ contains a cycle, and all its cycles have at least two adjacent anti-dependency edges.
\end{frame}

\begin{frame}
\begin{center}
	Thank you!
\end{center}
\end{frame}


\end{document}


